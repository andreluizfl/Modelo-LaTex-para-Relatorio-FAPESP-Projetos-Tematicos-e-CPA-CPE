%%-----
%% Página de título
%% Observação: As definições que aparecem a seguir comporão a
%%             página de título e a folha de rosto.
%%-----
%% Define o nome da universidade onde o projeto foi desenvolvido.
\LARGE
\universidade{<Nome da Universidade>}
%
%% Define o nome do departamento onde o projeto foi desenvolvido.
\instituicao{<Nome do Instituto ou Departamento>}
%
\normalsize
%% Define o título do projeto.
\titulo{<Nome do Projeto Temático ou CPA/CPE>}
%
%% Define a agencia de Fomento e a abreviatura. O primeiro argumento é o 
%% nome por extenso e o segundo a abreviatura.
%% Ambos os argumentos são obrigatórios
\agFomento{Fundação de Amparo à Pesquisa do Estado de São Paulo}{FAPESP}
%
%% Define o tipo de relatório. Pode ser Anual, Final ou Enumerado.
%% Não é obrigatório definir o tipo de relatório.
\tipoRelatorio{1}
%
%% Define a modalidade de Projeto. Pode ser temático, regular, etc.
\modalidadeProjeto{<Modalidade: Projeto Temático ou CPA/CPE>}
%
%% Define o número do projeto.
%% Não é obrigatório definir o número do projeto.
\numProjeto{YYYY/XXXXXX-D} 
%
%% Define o coordenador do projeto responsável pelo relatório.
\coordenador{<Nome do Coordenador>}
\instCoordenador{<Nome da Instituição (sigla)>}

%
%% Define a cidade onde o projeto foi desenvolvido.
\cidade{São Carlos}